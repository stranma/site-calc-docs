\section{Functional Requirements}

This section describes the functional requirements for the Site Energy System Optimization Platform, progressing from high-level capabilities to detailed functional specifications. Requirements are organized by major subsystem: optimization engine, input data management, economic analysis, results and reporting, REST API, Python client library, and web interface.

\subsection{Optimization Engine}

The optimization engine is the core computational component that determines optimal technology sizing and operational schedules. The engine must support multi-material flow optimization, handle both sizing and dispatch decisions simultaneously, and provide solver-independent operation.

\subsubsection{Technology Sizing Optimization}

\textbf{FR-OPT-001: Variable Capacity Optimization}

The system shall treat technology capacities as optimization decision variables, not fixed inputs. For each supported device type, the optimizer shall determine economically optimal sizing within user-specified bounds.

Supported sizing variables:
\begin{itemize}
\item Photovoltaic capacity (kWp): 0 to maximum site constraint
\item Battery energy capacity (kWh): 0 to maximum site constraint
\item Battery power rating (kW): 0 to maximum site constraint
\item CHP electrical capacity (MW): 0 to maximum site constraint
\item Heat accumulator thermal capacity (kWh\textsubscript{th}): 0 to maximum site constraint
\item Grid connection reserved power (kW): minimum contractual to maximum physical
\end{itemize}

Users shall specify bounds for each variable. The optimizer returns optimal values or boundary values when constraints are active.

\textbf{FR-OPT-002: Economies of Scale Modeling}

Investment costs shall reflect realistic economies of scale using power-law relationships: $C = a \times P^b$ where $b < 1$. Users specify cost coefficients $(a, b)$ for each technology. This ensures larger installations receive appropriate per-unit cost reductions.

\textbf{FR-OPT-003: Size-Power Decoupling}

For battery storage, energy capacity (kWh) and power rating (kW) shall be independently optimizable. This enables the system to find optimal C-rates (power-to-capacity ratios) for different applications: high C-rates for peak shaving (1C-2C), low C-rates for energy arbitrage (0.25C-0.5C).

\subsubsection{Operational Optimization}

\textbf{FR-OPT-004: Hourly Dispatch Scheduling}

The system shall optimize operational schedules with hourly time resolution over annual time horizons (8,760 hours). Dispatch decisions include:
\begin{itemize}
\item Battery charge/discharge power each hour
\item CHP electrical output each hour
\item Grid import/export each hour
\item Heat accumulator charge/discharge each hour
\item Grid service capacity reservation each hour (aFRR)
\end{itemize}

\textbf{FR-OPT-005: Multi-Material Flow Balance}

The system shall maintain energy balance constraints for all material types every hour:
\begin{itemize}
\item Electricity: generation + storage discharge + grid import = consumption + storage charge + grid export
\item Heat: CHP heat output + accumulator discharge + boiler = heat demand + accumulator charge + losses
\item Gas: grid supply = CHP consumption + boiler consumption
\end{itemize}

Flow balance must account for conversion efficiencies and losses.

\textbf{FR-OPT-006: Storage State Dynamics}

For all storage devices (battery, heat accumulator), the system shall model state-of-charge dynamics:
\begin{itemize}
\item State evolution: $\text{SOC}_{t+1} = \text{SOC}_t + \eta_{\text{charge}} P_{\text{charge},t} \Delta t - P_{\text{discharge},t} / \eta_{\text{discharge}} \Delta t - P_{\text{loss},t} \Delta t$
\item State bounds: $\text{SOC}_{\text{min}} \leq \text{SOC}_t \leq \text{SOC}_{\text{max}}$
\item Cyclic constraint: $\text{SOC}_{t=0} = \text{SOC}_{t=8760}$ (storage returns to initial state)
\item No simultaneous charge/discharge (binary constraints or continuous with losses)
\end{itemize}

\textbf{FR-OPT-007: Device Operational Constraints}

The system shall enforce technology-specific operational limits:

\textit{Battery constraints:}
\begin{itemize}
\item Power limits: $P_{\text{charge/discharge}} \leq P_{\text{rated}}$
\item C-rate limits: $P / E \leq C_{\text{max}}$ (typically 1-2C)
\item Depth of discharge: SOC $\geq$ minimum (typically 10-20\%)
\end{itemize}

\textit{CHP constraints:}
\begin{itemize}
\item Minimum stable output: $P_{\text{el}} \geq 0.4 \times P_{\text{rated}}$ when operating
\item Heat-to-power coupling: $P_{\text{heat}} = \alpha \times P_{\text{el}}$ where $\alpha$ is heat-to-power ratio
\item Part-load efficiency degradation (optional, for advanced modeling)
\end{itemize}

\textit{Grid connection constraints:}
\begin{itemize}
\item Import limit: $P_{\text{import}} \leq P_{\text{reserved}}$
\item Export limit: $P_{\text{export}} \leq P_{\text{export,max}}$ (may differ from import)
\end{itemize}

\subsubsection{Grid Service Integration}

\textbf{FR-OPT-008: aFRR Capacity Reservation}

The system shall model participation in automatic frequency restoration reserve (aFRR) markets:
\begin{itemize}
\item Minimum bid size: 1 MW (Czech market requirement)
\item Hourly capacity reservation decisions (how much battery capacity to offer)
\item Capacity availability constraint: reserved capacity unavailable for facility energy management
\item State-of-charge management: maintain SOC near 50\% to provide symmetric response
\item Revenue calculation: capacity payment (EUR/MW/hour) + estimated energy payment
\end{itemize}

\textbf{FR-OPT-009: Peak Shaving Optimization}

The system shall optimize peak demand reduction to minimize reserved capacity fees:
\begin{itemize}
\item Track maximum monthly peak demand across all hours in each month
\item Annual reserved capacity is maximum of monthly peaks
\item Coordinate battery discharge, PV generation, and CHP operation to reduce peaks
\item Balance peak shaving benefit (capacity fee reduction) against energy cost
\end{itemize}

\subsubsection{Objective Functions}

\textbf{FR-OPT-010: Economic Objectives}

The system shall support the following objective functions:
\begin{itemize}
\item \textbf{Minimize Total Cost}: Minimize CAPEX + NPV(OPEX - Revenue) over project horizon
\item \textbf{Maximize NPV}: Maximize net present value at specified discount rate
\item \textbf{Maximize IRR}: Find configuration maximizing internal rate of return (requires iterative optimization)
\item \textbf{Minimize Payback Period}: Find configuration with shortest payback (requires iterative optimization)
\end{itemize}

The primary objective is cost minimization / NPV maximization, which can be solved directly. IRR and payback optimization require wrapper algorithms calling the optimizer iteratively.

\textbf{FR-OPT-011: Constraint Handling}

The system shall return meaningful results when constraints conflict:
\begin{itemize}
\item Infeasible problems: report which constraints conflict
\item Unbounded problems: report which variables lack upper bounds
\item Suboptimal solutions: report optimality gap when time limits are reached
\end{itemize}

\subsubsection{Solver Architecture}

\textbf{FR-OPT-012: Solver Independence}

The optimization engine shall support multiple solver backends without requiring changes to problem formulation:
\begin{itemize}
\item \textbf{CBC}: Open-source MILP solver (default, always available)
\item \textbf{HiGHS}: High-performance open-source LP/MILP solver
\item \textbf{Gurobi}: Commercial solver (optional, when license available)
\item \textbf{CPLEX}: Commercial solver (optional, when license available)
\end{itemize}

Users select solver at runtime. The system automatically translates domain model to solver-specific format.

\textbf{FR-OPT-013: Performance Requirements}

The system shall solve typical problems within reasonable timeframes:
\begin{itemize}
\item Small problems (1-2 devices, no MILP): $< 30$ seconds
\item Medium problems (3-4 devices, simple MILP): $< 5$ minutes
\item Large problems (5+ devices, complex MILP): $< 30$ minutes
\end{itemize}

Times assume open-source solvers (CBC, HiGHS). Commercial solvers may achieve 10-100x speedup.

\subsection{Input Data Management}

The system requires three categories of input data: load profiles, price forecasts, and technology parameters. Input handling must support multiple data formats, validate data quality, and integrate external data sources where appropriate.

\subsubsection{Load Profiles}

\textbf{FR-INPUT-001: Electricity Load Profile}

The system shall accept annual hourly electricity consumption profiles (8,760 values):
\begin{itemize}
\item Format: CSV, Excel, JSON, or API submission
\item Units: kW (instantaneous power) or kWh (hourly energy)
\item Validation: positive values, no missing hours, reasonable magnitude (compared to stated facility size)
\item Temporal alignment: must specify timezone and handling of daylight saving time
\end{itemize}

For facilities with interval metering data, users upload directly. For new facilities, users may provide synthetic profiles based on facility type and size.

\textbf{FR-INPUT-002: Heat Demand Profile}

For facilities with thermal loads (CHP optimization), the system shall accept hourly heat demand:
\begin{itemize}
\item Format: CSV, Excel, JSON, or API submission
\item Units: kW\textsubscript{th} or kWh\textsubscript{th}
\item Validation: positive values, no missing hours, consistent with facility thermal requirements
\item Seasonal patterns: system should warn if heat profile lacks expected seasonal variation
\end{itemize}

\textbf{FR-INPUT-003: Load Profile Scaling}

The system shall support scaling reference load profiles:
\begin{itemize}
\item Provide single year of hourly data
\item System replicates pattern for multi-year optimizations
\item Optional: specify year-over-year growth rate (e.g., 2\% annual increase)
\end{itemize}

\subsubsection{Price Forecasts}

\textbf{FR-INPUT-004: Electricity Spot Prices}

The system shall accept hourly electricity price forecasts:
\begin{itemize}
\item Format: CSV, Excel, JSON, or API submission
\item Units: EUR/MWh or CZK/MWh
\item Duration: single year (replicated) or full project horizon (8,760 to 175,200 hours for 20 years)
\item Validation: reasonable price ranges (negative allowed for renewable curtailment scenarios)
\end{itemize}

For multi-year forecasts, users may provide year-specific profiles or single year with escalation rate.

\textbf{FR-INPUT-005: Grid Tariffs}

The system shall accept Czech distribution tariff structure:
\begin{itemize}
\item \textbf{Reserved capacity fee}: CZK/kW/month (applied to monthly peak or contracted capacity)
\item \textbf{Distribution charge}: CZK/kWh (may be time-of-use with high/low rates)
\item \textbf{Electricity tax}: CZK/MWh (currently 28.30 CZK/MWh)
\item \textbf{System services charge}: CZK/MWh
\end{itemize}

Tariffs may be specified as constants or time series (for modeling tariff evolution).

\textbf{FR-INPUT-006: Fuel Prices}

For CHP optimization, the system shall accept natural gas prices:
\begin{itemize}
\item Units: EUR/MWh\textsubscript{th} (thermal)
\item Duration: single value or time series (monthly/annual variation)
\item Validation: realistic ranges (20-80 EUR/MWh\textsubscript{th} typical)
\end{itemize}

\textbf{FR-INPUT-007: Grid Service Prices}

For aFRR optimization, the system shall accept market price forecasts:
\begin{itemize}
\item \textbf{Capacity payment}: EUR/MW/hour (payment for making capacity available)
\item \textbf{Energy payment}: EUR/MWh (payment when activated) or percentage premium over spot price
\item Duration: hourly time series or statistical distribution
\end{itemize}

\subsubsection{Technology Parameters}

\textbf{FR-INPUT-008: Photovoltaic Parameters}

Users shall specify PV system characteristics:
\begin{itemize}
\item \textbf{Maximum capacity}: kWp (upper bound for optimization)
\item \textbf{Capital cost}: EUR/kWp (linear) or power-law coefficients $(a, b)$
\item \textbf{Fixed O\&M}: EUR/kWp/year
\item \textbf{Degradation}: \%/year (typically 0.5\%)
\item \textbf{Lifetime}: years (typically 25-30)
\item \textbf{Location}: latitude, longitude for PVGIS integration
\item \textbf{Mounting}: fixed tilt/azimuth or tracking system
\end{itemize}

\textbf{FR-INPUT-009: Battery Parameters}

Users shall specify battery storage characteristics:
\begin{itemize}
\item \textbf{Maximum energy capacity}: kWh
\item \textbf{Maximum power rating}: kW
\item \textbf{Capital cost}: EUR/kWh (energy) + EUR/kW (power), or combined power-law
\item \textbf{Round-trip efficiency}: \% (typically 85-95\%)
\item \textbf{Self-discharge}: \%/day (typically 0.1-0.5\%)
\item \textbf{Depth-of-discharge limits}: \% (typically 10-90\%)
\item \textbf{Degradation model}: cycle-based or throughput-based
\item \textbf{Replacement schedule}: years until replacement (typically 10-15)
\item \textbf{Fixed O\&M}: EUR/kWh/year
\end{itemize}

\textbf{FR-INPUT-010: CHP Parameters}

Users shall specify CHP unit characteristics:
\begin{itemize}
\item \textbf{Maximum electrical capacity}: MW\textsubscript{el}
\item \textbf{Electrical efficiency}: \% (typically 35-45\%)
\item \textbf{Thermal efficiency}: \% (typically 40-50\%)
\item \textbf{Heat-to-power ratio}: dimensionless (typically 0.9-1.3)
\item \textbf{Capital cost}: EUR/kW\textsubscript{el} with economies of scale
\item \textbf{Variable O\&M}: EUR/MWh\textsubscript{el}
\item \textbf{Fixed O\&M}: EUR/kW\textsubscript{el}/year
\item \textbf{Minimum part-load}: \% (typically 40-50\%)
\item \textbf{Lifetime}: hours (typically 60,000-80,000 hours)
\end{itemize}

\textbf{FR-INPUT-011: Heat Accumulator Parameters}

Users shall specify thermal storage characteristics:
\begin{itemize}
\item \textbf{Maximum capacity}: kWh\textsubscript{th}
\item \textbf{Capital cost}: EUR/kWh\textsubscript{th} with economies of scale
\item \textbf{Heat loss rate}: \%/day (typically 1-3\%)
\item \textbf{Charge/discharge efficiency}: \% (typically 95-98\%)
\item \textbf{Maximum charge/discharge power}: kW\textsubscript{th}
\item \textbf{Fixed O\&M}: EUR/kWh\textsubscript{th}/year
\end{itemize}

\subsubsection{External Data Integration}

\textbf{FR-INPUT-012: PVGIS Integration}

The system shall automatically retrieve PV generation profiles from PVGIS:
\begin{itemize}
\item Input: latitude, longitude, panel specifications (tilt, azimuth, technology)
\item Output: hourly generation profile (kWh per kWp installed)
\item Validation: check PVGIS service availability, handle API errors gracefully
\item Caching: cache PVGIS results to avoid repeated API calls for same location
\end{itemize}

This eliminates manual preparation of weather-dependent generation data.

\textbf{FR-INPUT-013: Data Validation and Quality Checks}

The system shall validate all input data:
\begin{itemize}
\item \textbf{Completeness}: no missing hours in time series
\item \textbf{Ranges}: values within physically realistic bounds
\item \textbf{Consistency}: e.g., heat demand compatible with CHP sizing
\item \textbf{Units}: clear specification and automatic conversion
\item \textbf{Timezone}: explicit timezone handling for hourly data
\end{itemize}

Validation failures shall produce clear error messages identifying the problem.

\subsection{Economic Analysis}

Economic analysis converts optimization results into financial metrics suitable for investment decisions. The system must model all cost and revenue streams over multi-decade project horizons with appropriate discounting.

\subsubsection{Investment Costs (CAPEX)}

\textbf{FR-ECON-001: Equipment Costs with Economies of Scale}

The system shall calculate equipment costs using power-law relationships:
\[
C_{\text{equipment}} = a \times P^b
\]
where $P$ is capacity (kW, kWh, kWp), $a$ is the cost coefficient, and $b < 1$ represents economies of scale. Typical values: $b = 0.7$-$0.9$ depending on technology.

\textbf{FR-ECON-002: Installation and Commissioning}

Users may specify installation costs as:
\begin{itemize}
\item Percentage of equipment cost (e.g., 15-25\%)
\item Fixed amount (EUR)
\item Detailed breakdown by labor, materials, permits
\end{itemize}

\textbf{FR-ECON-003: Grid Connection Costs}

If optimized reserved capacity exceeds existing contract, system calculates grid connection upgrade costs:
\begin{itemize}
\item Fixed cost per kW of capacity increase
\item May include transformer upgrades, substation work
\item Users specify cost structure for their distribution zone
\end{itemize}

\textbf{FR-ECON-004: Total CAPEX Calculation}

Total initial investment:
\[
\text{CAPEX} = \sum_{\text{devices}} C_{\text{equipment}} + C_{\text{installation}} + C_{\text{grid}} + C_{\text{development}}
\]

\subsubsection{Operating Costs (OPEX)}

\textbf{FR-ECON-005: Fixed O\&M Costs}

Annual fixed operating costs proportional to installed capacity:
\[
C_{\text{fixed,annual}} = \sum_{\text{devices}} c_{\text{fixed}} \times P_{\text{installed}}
\]
Units: EUR/kW/year or EUR/kWh/year depending on device type.

\textbf{FR-ECON-006: Variable O\&M Costs}

Operating costs proportional to energy throughput:
\[
C_{\text{variable,annual}} = \sum_{\text{devices}} c_{\text{variable}} \times E_{\text{annual}}
\]
Units: EUR/MWh. Important for CHP (maintenance per hour of operation).

\textbf{FR-ECON-007: Fuel Costs}

For CHP systems, annual fuel costs:
\[
C_{\text{fuel,annual}} = \sum_{t=1}^{8760} P_{\text{CHP},t} \times \Delta t / \eta_{\text{el}} \times p_{\text{gas},t}
\]
where $\eta_{\text{el}}$ is electrical efficiency and $p_{\text{gas}}$ is gas price.

\textbf{FR-ECON-008: Grid Charges}

\textit{Reserved capacity fee (Czech specific):}
\[
C_{\text{capacity}} = P_{\text{reserved}} \times p_{\text{capacity}} \times 12
\]
Units: CZK/kW/month converted to annual cost. Typical range: 200-400 CZK/kW/month.

\textit{Distribution tariff:}
\[
C_{\text{distribution}} = \sum_{t=1}^{8760} E_{\text{import},t} \times p_{\text{distribution},t}
\]

\textit{Electricity tax and system services:}
Fixed CZK/MWh charges applied to all consumption.

\textbf{FR-ECON-009: Degradation and Replacement Reserves}

The system shall model equipment replacement:
\begin{itemize}
\item \textbf{Battery replacement}: typically year 10-15, cost 70\% of initial (excluding power electronics)
\item \textbf{PV inverter replacement}: typically year 12-15, cost 10-15\% of initial system cost
\item \textbf{CHP major overhaul}: based on operating hours, cost 20-30\% of initial
\end{itemize}

Replacement costs are discounted to present value.

\subsubsection{Revenue Streams}

\textbf{FR-ECON-010: Energy Cost Savings}

Primary revenue is avoided grid purchase cost:
\[
R_{\text{savings}} = \sum_{t=1}^{8760} E_{\text{generation},t} \times (p_{\text{electricity},t} + p_{\text{distribution},t} + p_{\text{tax}})
\]

This represents energy generated on-site that displaces grid purchase.

\textbf{FR-ECON-011: Electricity Sales}

If facility exports to grid (PV oversizing):
\[
R_{\text{export}} = \sum_{t=1}^{8760} E_{\text{export},t} \times p_{\text{export},t}
\]

Note: export prices typically lower than import prices (no distribution charge refund).

\textbf{FR-ECON-012: aFRR Revenues}

\textit{Capacity payment:}
\[
R_{\text{aFRR,capacity}} = \sum_{t=1}^{8760} P_{\text{aFRR},t} \times p_{\text{aFRR,capacity},t}
\]
Units: EUR/MW/hour.

\textit{Energy payment (estimated):}
\[
R_{\text{aFRR,energy}} = \sum_{t=1}^{8760} P_{\text{aFRR},t} \times f_{\text{activation}} \times p_{\text{aFRR,energy},t}
\]
where $f_{\text{activation}}$ is expected activation rate (e.g., 10-20\%).

\textbf{FR-ECON-013: Peak Demand Charge Reduction}

Savings from reduced reserved capacity:
\[
R_{\text{peak}} = (P_{\text{baseline}} - P_{\text{optimized}}) \times p_{\text{capacity}} \times 12
\]

This is a major value stream for industrial facilities with high capacity charges.

\subsubsection{Financial Metrics}

\textbf{FR-ECON-014: Net Present Value (NPV)}

Calculate NPV over project horizon:
\[
\text{NPV} = -\text{CAPEX} + \sum_{y=1}^{N} \frac{CF_y}{(1+r)^y}
\]
where $CF_y$ is annual cash flow (revenues minus OPEX) in year $y$, $r$ is discount rate, and $N$ is project lifetime (typically 20 years).

\textbf{FR-ECON-015: Internal Rate of Return (IRR)}

Calculate IRR as the discount rate where NPV = 0:
\[
0 = -\text{CAPEX} + \sum_{y=1}^{N} \frac{CF_y}{(1+\text{IRR})^y}
\]

Solved numerically. Report if no solution exists (project never profitable).

\textbf{FR-ECON-016: Return on Investment (ROI)}

Simple ROI calculation:
\[
\text{ROI} = \frac{\sum_{y=1}^{N} CF_y - \text{CAPEX}}{\text{CAPEX}} \times 100\%
\]

Also calculate annualized ROI: $\text{ROI}_{\text{annual}} = \text{ROI} / N$.

\textbf{FR-ECON-017: Payback Period}

\textit{Simple payback:}
Minimum $y$ where $\sum_{i=1}^{y} CF_i \geq \text{CAPEX}$.

\textit{Discounted payback:}
Minimum $y$ where $\sum_{i=1}^{y} \frac{CF_i}{(1+r)^i} \geq \text{CAPEX}$.

\textbf{FR-ECON-018: Levelized Cost of Energy (LCOE)}

For generation technologies (PV, CHP):
\[
\text{LCOE} = \frac{\text{CAPEX} + \sum_{y=1}^{N} \frac{\text{OPEX}_y}{(1+r)^y}}{\sum_{y=1}^{N} \frac{E_y}{(1+r)^y}}
\]
Units: EUR/MWh. Useful for comparing generation technologies.

\subsubsection{Sensitivity Analysis}

\textbf{FR-ECON-019: Single-Parameter Sensitivity}

The system shall automatically generate sensitivity analysis varying key parameters:
\begin{itemize}
\item Electricity price: $\pm 20\%$, $\pm 50\%$
\item Technology costs: $\pm 20\%$, $\pm 50\%$
\item Discount rate: $\pm 2$ percentage points
\item Degradation rate: $\pm 50\%$
\end{itemize}

Report impact on NPV, IRR, optimal sizing.

\textbf{FR-ECON-020: Scenario Comparison}

Users shall compare multiple technology configurations with identical economic assumptions:
\begin{itemize}
\item Scenario 1: PV only
\item Scenario 2: PV + Battery
\item Scenario 3: PV + Battery + aFRR
\item Scenario 4: CHP + Heat Accumulator
\end{itemize}

Present side-by-side comparison table with NPV, IRR, ROI, sizing, CAPEX.

\subsection{Results and Reporting}

The system must present results clearly for both technical analysis and business decision-making. Primary outputs focus on optimal sizing and economic metrics; operational details are secondary.

\subsubsection{Optimal Sizing Results}

\textbf{FR-RESULT-001: Technology Sizing Summary}

For each optimized technology, report:
\begin{itemize}
\item Optimal capacity (kW, kWh, kWp)
\item Boundary status (at lower bound, interior solution, at upper bound)
\item Shadow price (marginal value of relaxing capacity constraint)
\end{itemize}

Example output:
\begin{itemize}
\item PV: 220 kWp (interior solution, shadow price: 50 EUR/kWp)
\item Battery Power: 120 kW (at upper bound, shadow price: 120 EUR/kW)
\item Battery Energy: 480 kWh (interior solution, shadow price: 35 EUR/kWh)
\end{itemize}

\textbf{FR-RESULT-002: Grid Connection Sizing}

Report optimized reserved capacity:
\begin{itemize}
\item Optimized reserved capacity (kW)
\item Baseline reserved capacity without optimization (kW)
\item Reduction achieved (kW and \%)
\item Annual capacity fee savings (EUR/year)
\end{itemize}

\subsubsection{Economic Results}

\textbf{FR-RESULT-003: Economic Summary Table}

Present financial metrics in structured format:

\begin{table}[h]
\centering
\begin{tabular}{lr}
\toprule
\textbf{Metric} & \textbf{Value} \\
\midrule
Total CAPEX & 330,000 EUR \\
Annual OPEX & 8,500 EUR/year \\
Annual Revenue & 68,500 EUR/year \\
Annual Net Cash Flow & 60,000 EUR/year \\
\midrule
NPV (6\% discount) & 250,000 EUR \\
IRR & 11.2\% \\
ROI & 263\% \\
Simple Payback & 5.5 years \\
Discounted Payback & 8.1 years \\
\bottomrule
\end{tabular}
\end{table}

\textbf{FR-RESULT-004: Cash Flow Projection}

Provide annual cash flow table for project lifetime:
\begin{itemize}
\item Year 0: CAPEX (negative)
\item Years 1-20: annual revenues, OPEX, net cash flow
\item Years with major replacements (e.g., battery replacement year 10)
\item Cumulative cash flow (for payback visualization)
\end{itemize}

\textbf{FR-RESULT-005: Cost and Revenue Breakdown}

Detailed breakdown of annual economics:

\textit{OPEX Components:}
\begin{itemize}
\item Fixed O\&M by device
\item Variable O\&M by device
\item Fuel costs (CHP)
\item Grid charges (capacity fee, distribution, taxes)
\end{itemize}

\textit{Revenue Components:}
\begin{itemize}
\item Energy cost savings
\item Electricity export revenue
\item aFRR capacity revenue
\item aFRR energy revenue (estimated)
\item Peak demand charge reduction
\end{itemize}

\subsubsection{Operational Profiles (Secondary)}

\textbf{FR-RESULT-006: Typical Day Visualizations}

Generate operational profile visualizations for representative days:
\begin{itemize}
\item Winter day (high load, low PV)
\item Summer day (moderate load, high PV)
\item Shoulder season day
\end{itemize}

Each visualization shows hourly time series:
\begin{itemize}
\item Electricity load (kW)
\item PV generation (kW)
\item Battery charge/discharge (kW)
\item Grid import/export (kW)
\item Battery state-of-charge (\%)
\item Electricity price (EUR/MWh) as background color
\end{itemize}

\textbf{FR-RESULT-007: Annual Energy Balance}

Summary statistics over full year:
\begin{itemize}
\item Total load (MWh)
\item Total PV generation (MWh)
\item Total battery throughput (MWh)
\item Total grid import (MWh)
\item Total grid export (MWh)
\item Self-consumption rate (\%)
\item Battery utilization (cycles per year)
\end{itemize}

\textbf{FR-RESULT-008: Peak Demand Analysis}

Monthly peak demand table:
\begin{itemize}
\item Month-by-month maximum demand (kW)
\item Baseline maximum demand (kW)
\item Peak reduction achieved (kW)
\item Date and hour of peak occurrence
\end{itemize}

\subsubsection{Comparison and Sensitivity Results}

\textbf{FR-RESULT-009: Scenario Comparison Matrix}

Side-by-side comparison table for multiple scenarios:

\begin{table}[h]
\centering
\small
\begin{tabular}{lrrr}
\toprule
\textbf{Metric} & \textbf{PV Only} & \textbf{PV + Battery} & \textbf{PV + Bat + aFRR} \\
\midrule
PV Capacity (kWp) & 250 & 220 & 220 \\
Battery (kW / kWh) & --- & 120 / 480 & 150 / 600 \\
CAPEX (EUR) & 200,000 & 330,000 & 380,000 \\
Annual Cash Flow (EUR) & 35,000 & 60,000 & 75,000 \\
NPV (EUR) & 180,000 & 250,000 & 310,000 \\
IRR (\%) & 10.5 & 11.2 & 12.8 \\
Payback (years) & 5.7 & 8.1 & 7.4 \\
\bottomrule
\end{tabular}
\end{table}

\textbf{FR-RESULT-010: Sensitivity Charts}

Graphical sensitivity analysis:
\begin{itemize}
\item Tornado diagram: ranking parameters by NPV impact
\item NPV vs. electricity price curve
\item NPV vs. discount rate curve
\item IRR vs. technology cost curve
\end{itemize}

\subsubsection{Export Formats}

\textbf{FR-RESULT-011: PDF Report Generation}

Automated PDF report including:
\begin{itemize}
\item Executive summary (1 page)
\item Economic results (tables and charts)
\item Optimal sizing recommendations
\item Sensitivity analysis
\item Sample operational profiles (selected days)
\item Assumptions and input data summary
\end{itemize}

Professional formatting suitable for client presentations and loan applications.

\textbf{FR-RESULT-012: Excel Export}

Detailed Excel workbook with multiple sheets:
\begin{itemize}
\item Summary: economic metrics and sizing
\item Hourly Results: complete 8,760-hour operational schedule
\item Cash Flow: annual cash flow projection
\item Sensitivity: parameter variation results
\item Input Data: echo of all input parameters for reproducibility
\end{itemize}

\textbf{FR-RESULT-013: API JSON Response}

For programmatic access, structured JSON containing all results:
\begin{itemize}
\item Optimal sizing (machine-readable)
\item Economic metrics (with units)
\item Hourly operational data (optional, can be large)
\item Metadata: optimization runtime, solver used, solution status
\end{itemize}

\subsection{REST API}

The REST API provides programmatic access to the optimization platform, enabling integration with external workflows, Monte Carlo frameworks, and custom applications.

\subsubsection{API Architecture}

\textbf{FR-API-001: RESTful Design}

API shall follow REST principles:
\begin{itemize}
\item Resource-based URLs: \texttt{/projects}, \texttt{/optimizations}, \texttt{/results}
\item HTTP methods: GET (retrieve), POST (create), PUT (update), DELETE (remove)
\item Stateless: each request contains all necessary information
\item JSON request/response bodies
\end{itemize}

\textbf{FR-API-002: OpenAPI Documentation}

Complete API documentation using OpenAPI 3.0 specification:
\begin{itemize}
\item Interactive Swagger UI for testing endpoints
\item Auto-generated client libraries in multiple languages
\item Schema definitions for all request/response objects
\item Example requests and responses
\end{itemize}

\textbf{FR-API-003: Authentication}

API key-based authentication:
\begin{itemize}
\item Each user/organization receives API key
\item Key included in HTTP header: \texttt{Authorization: Bearer <token>}
\item Rate limiting per API key
\item Key rotation support
\end{itemize}

\subsubsection{Core Endpoints}

\textbf{FR-API-004: Project Management}

\texttt{POST /projects} - Create new project
\begin{itemize}
\item Request: project name, description, site location
\item Response: project ID, creation timestamp
\end{itemize}

\texttt{GET /projects} - List user's projects

\texttt{GET /projects/\{id\}} - Get project details

\texttt{DELETE /projects/\{id\}} - Delete project

\textbf{FR-API-005: Optimization Job Submission}

\texttt{POST /projects/\{id\}/optimizations} - Submit optimization job
\begin{itemize}
\item Request body includes:
  \begin{itemize}
  \item Load profiles (electricity, heat)
  \item Price forecasts (electricity, gas, grid services)
  \item Technology specifications (devices to optimize)
  \item Economic parameters (discount rate, project lifetime)
  \item Optimization settings (objective, solver preference)
  \end{itemize}
\item Response: optimization job ID, status (queued)
\end{itemize}

\texttt{GET /optimizations/\{id\}} - Get optimization status
\begin{itemize}
\item Response: status (queued | running | completed | failed), progress percentage
\end{itemize}

\textbf{FR-API-006: Results Retrieval}

\texttt{GET /optimizations/\{id\}/results} - Get optimization results
\begin{itemize}
\item Response includes:
  \begin{itemize}
  \item Optimal sizing
  \item Economic metrics
  \item Solution status (optimal | suboptimal | infeasible)
  \end{itemize}
\end{itemize}

\texttt{GET /optimizations/\{id\}/results/hourly} - Get detailed hourly data

\texttt{GET /optimizations/\{id\}/results/export?format=excel} - Export results

\textbf{FR-API-007: Scenario Comparison}

\texttt{POST /projects/\{id\}/comparisons} - Create scenario comparison
\begin{itemize}
\item Request: list of optimization IDs to compare
\item Response: comparison matrix with side-by-side results
\end{itemize}

\subsubsection{Data Upload Endpoints}

\textbf{FR-API-008: Time Series Upload}

\texttt{POST /projects/\{id\}/data/load-profile} - Upload load profile CSV

\texttt{POST /projects/\{id\}/data/prices} - Upload price forecast CSV

\textbf{FR-API-009: PVGIS Integration Endpoint}

\texttt{POST /projects/\{id\}/data/pvgis} - Request PV generation profile
\begin{itemize}
\item Request: latitude, longitude, panel specifications
\item Response: hourly generation profile (kWh per kWp)
\item Caches result for this location/configuration
\end{itemize}

\subsubsection{Asynchronous Operation}

\textbf{FR-API-010: Long-Running Jobs}

Optimizations are asynchronous:
\begin{enumerate}
\item Client POSTs optimization job → receives job ID
\item Client polls \texttt{GET /optimizations/\{id\}} for status
\item When status = completed, client GETs results
\end{enumerate}

\textbf{FR-API-011: Webhook Notifications}

Optional webhook support:
\begin{itemize}
\item User specifies webhook URL when submitting optimization
\item System POSTs to webhook when optimization completes
\item Webhook payload includes job ID and status
\end{itemize}

\subsubsection{Error Handling}

\textbf{FR-API-012: HTTP Status Codes}

Standard HTTP status codes:
\begin{itemize}
\item 200 OK: successful GET
\item 201 Created: successful POST
\item 400 Bad Request: invalid input data
\item 401 Unauthorized: missing/invalid API key
\item 404 Not Found: resource doesn't exist
\item 429 Too Many Requests: rate limit exceeded
\item 500 Internal Server Error: server-side failure
\end{itemize}

\textbf{FR-API-013: Error Response Format}

Structured error responses:
\begin{verbatim}
{
  "error": {
    "code": "INVALID_INPUT",
    "message": "Load profile contains negative values",
    "details": {
      "field": "load_profile",
      "invalid_hours": [523, 524, 525]
    }
  }
}
\end{verbatim}

\subsection{Python Client Library}

The Python client library provides a high-level, Pythonic interface to the REST API, abstracting device models and making integration seamless for data science workflows.

\subsubsection{Package Structure}

\textbf{FR-PY-001: Package Distribution}

\begin{itemize}
\item Package name: \texttt{site-calc-client}
\item Distribution: PyPI (\texttt{pip install site-calc-client})
\item Python version: 3.9+
\item Dependencies: \texttt{requests}, \texttt{pandas}, \texttt{pydantic}
\end{itemize}

\textbf{FR-PY-002: Auto-Generation from OpenAPI}

Client library auto-generated from OpenAPI specification ensures:
\begin{itemize}
\item API compatibility guaranteed
\item Automatic updates when API changes
\item Consistent error handling
\end{itemize}

High-level device abstractions hand-coded for better user experience.

\subsubsection{Core API}

\textbf{FR-PY-003: Client Initialization}

\begin{verbatim}
from site_calc_client import Client

client = Client(api_key="your-api-key")
# Or use environment variable: SITE_CALC_API_KEY
client = Client()
\end{verbatim}

\textbf{FR-PY-004: Site and Device Configuration}

\begin{verbatim}
from site_calc_client import Site, Battery, PV

site = Site(name="Industrial Facility")

# Add devices with high-level parameters
site.add_device(PV(
    name="Rooftop PV",
    max_capacity_kwp=500,
    location=(50.0755, 14.4378),  # Prague
    capex_per_kwp=800,
    opex_per_kwp_year=12
))

site.add_device(Battery(
    name="Main Battery",
    max_power_kw=250,
    max_capacity_kwh=1000,
    capex_per_kwh=200,
    capex_per_kw=150,
    efficiency=0.90,
    degradation_model="throughput"
))
\end{verbatim}

\textbf{FR-PY-005: Load and Price Data}

\begin{verbatim}
import pandas as pd

# Load profile from file
load_profile = pd.read_csv("load.csv", index_col=0)
site.set_load_profile(load_profile["electricity_kw"])

# Prices from DataFrame
prices = pd.read_csv("prices.csv", index_col=0)
site.set_electricity_prices(prices["spot_price_eur_mwh"])
site.set_grid_tariff(
    capacity_fee_czk_per_kw_month=300,
    distribution_czk_per_kwh=500
)
\end{verbatim}

\textbf{FR-PY-006: Optimization Execution}

\begin{verbatim}
# Configure optimization
optimization = client.create_optimization(
    site=site,
    objective="maximize_npv",
    time_horizon_years=20,
    discount_rate=0.06,
    solver="highs"  # optional, default: "cbc"
)

# Submit and wait for results
result = optimization.run()  # Blocking call
# Or async:
job = optimization.submit()  # Returns immediately
result = job.wait()  # Wait for completion
\end{verbatim}

\textbf{FR-PY-007: Results Access}

\begin{verbatim}
# Optimal sizing
print(f"PV: {result.get_device_capacity('Rooftop PV')} kWp")
print(f"Battery: {result.get_device_capacity('Main Battery')} kWh")
print(f"Battery Power: {result.get_device_power('Main Battery')} kW")

# Economic metrics
print(f"NPV: {result.npv:.0f} EUR")
print(f"IRR: {result.irr * 100:.1f}%")
print(f"Payback: {result.payback_years:.1f} years")

# Cash flow DataFrame
cash_flow = result.get_cash_flow_table()
# Returns: pandas DataFrame with columns [Year, OPEX, Revenue, Net, Cumulative]

# Hourly operational data
hourly = result.get_hourly_data()
# Returns: pandas DataFrame with hourly time series
\end{verbatim}

\subsubsection{Device Abstractions}

\textbf{FR-PY-008: Device Class Hierarchy}

Users work with concrete device types, not generic "Device":
\begin{itemize}
\item \texttt{PV(name, max\_capacity\_kwp, location, ...)}
\item \texttt{Battery(name, max\_power\_kw, max\_capacity\_kwh, ...)}
\item \texttt{CHP(name, max\_capacity\_mw, efficiency\_el, efficiency\_heat, ...)}
\item \texttt{HeatAccumulator(name, max\_capacity\_kwh\_th, ...)}
\end{itemize}

Each class has type-safe parameters with validation.

\textbf{FR-PY-009: No Internal Implementation Exposure}

Users do NOT see:
\begin{itemize}
\item Flow systems
\item Material balance equations
\item Port connections
\item Physics rules
\end{itemize}

Library presents intuitive energy engineering concepts only.

\subsubsection{Advanced Features}

\textbf{FR-PY-010: Scenario Comparison}

\begin{verbatim}
# Define multiple scenarios
scenario1 = client.create_optimization(
    site=site_pv_only,
    objective="maximize_npv",
    ...
)

scenario2 = client.create_optimization(
    site=site_pv_battery,
    objective="maximize_npv",
    ...
)

# Compare
comparison = client.compare_scenarios([scenario1, scenario2])
comparison.plot()  # Generate comparison charts
comparison.to_dataframe()  # Comparison matrix
\end{verbatim}

\textbf{FR-PY-011: Sensitivity Analysis}

\begin{verbatim}
sensitivity = optimization.sensitivity_analysis(
    parameter="electricity_price",
    range=(-0.5, 0.5),  # ±50%
    steps=10
)

sensitivity.plot()  # NPV vs. price chart
sensitivity.results  # DataFrame of results
\end{verbatim}

\textbf{FR-PY-012: Batch Processing}

For Monte Carlo integration:
\begin{verbatim}
results = []
for scenario in price_scenarios:  # 1000+ scenarios
    site.set_electricity_prices(scenario)
    opt = client.create_optimization(site=site, ...)
    result = opt.run()
    results.append({
        "npv": result.npv,
        "irr": result.irr,
        "pv_size": result.get_device_capacity("PV")
    })

results_df = pd.DataFrame(results)
results_df["npv"].describe()  # Statistical summary
\end{verbatim}

\subsubsection{Type Safety and Documentation}

\textbf{FR-PY-013: Type Hints}

All public methods fully typed:
\begin{verbatim}
def get_device_capacity(
    self,
    device_name: str
) -> float:
    """Get optimal capacity for named device.

    Args:
        device_name: Name of device (as specified in Site)

    Returns:
        Optimal capacity in device-specific units (kW, kWh, kWp)

    Raises:
        KeyError: If device_name not found in site
    """
\end{verbatim}

\textbf{FR-PY-014: Comprehensive Documentation}

\begin{itemize}
\item Docstrings for all classes and methods
\item Sphinx-generated HTML documentation
\item Jupyter notebook examples for common workflows
\item README with quickstart guide
\end{itemize}

\subsection{Web Interface}

The web interface provides interactive access for users who prefer graphical configuration over programmatic APIs. The interface must be intuitive for energy consultants without software engineering backgrounds.

\subsubsection{User Interface Structure}

\textbf{FR-WEB-001: Project Dashboard}

Landing page showing:
\begin{itemize}
\item List of user's projects with status
\item Recent optimizations (running, completed, failed)
\item Quick actions: create project, view results
\item Usage statistics (API calls, compute time)
\end{itemize}

\textbf{FR-WEB-002: Project Creation Wizard}

Step-by-step wizard for new projects:
\begin{enumerate}
\item \textbf{Basic Info}: project name, site location, facility type
\item \textbf{Load Data}: upload CSV or enter manually, validate time series
\item \textbf{Technology Selection}: checkboxes for PV, Battery, CHP, Heat Accumulator
\item \textbf{Technology Parameters}: forms for each selected technology
\item \textbf{Economic Assumptions}: discount rate, project lifetime, price forecasts
\item \textbf{Optimization Settings}: objective function, solver preference
\item \textbf{Review and Submit}: summary of configuration, submit button
\end{enumerate}

\textbf{FR-WEB-003: Data Upload Interface}

CSV/Excel file upload with preview:
\begin{itemize}
\item Drag-and-drop file upload
\item Column mapping interface (map CSV columns to expected fields)
\item Data validation with error highlighting
\item Chart preview of uploaded time series
\end{itemize}

\subsubsection{Technology Configuration}

\textbf{FR-WEB-004: Technology Selection}

Visual card-based technology selector:
\begin{itemize}
\item Card for each technology (PV, Battery, CHP, etc.) with icon and description
\item Toggle switch to enable/disable technology
\item When enabled, expand to show parameter form
\end{itemize}

\textbf{FR-WEB-005: Parameter Input Forms}

For each technology, structured form with:
\begin{itemize}
\item Required parameters highlighted
\item Optional parameters collapsed by default
\item Units clearly labeled
\item Validation (min/max ranges, required fields)
\item Tooltips explaining each parameter
\item Default values pre-filled (typical values)
\end{itemize}

\textbf{FR-WEB-006: PVGIS Integration UI}

For PV configuration:
\begin{itemize}
\item Interactive map for location selection (click to set lat/lon)
\item Or manual lat/lon entry
\item Button: "Fetch generation profile from PVGIS"
\item Progress indicator during API call
\item Preview chart of generation profile when loaded
\end{itemize}

\subsubsection{Optimization Execution}

\textbf{FR-WEB-007: Optimization Launch}

Submit page showing:
\begin{itemize}
\item Summary of configuration (all inputs)
\item Estimated solve time based on problem size
\item Solver selection dropdown (CBC, HiGHS, Gurobi if available)
\item "Start Optimization" button
\end{itemize}

\textbf{FR-WEB-008: Progress Monitoring}

Live status page during optimization:
\begin{itemize}
\item Progress bar (if solver reports progress)
\item Status messages: "Formulating problem", "Solving", "Extracting results"
\item Elapsed time
\item Option to cancel running optimization
\item Real-time log output (optional, for debugging)
\end{itemize}

\subsubsection{Results Visualization}

\textbf{FR-WEB-009: Results Summary Page}

High-level results overview:
\begin{itemize}
\item Hero metrics: NPV, IRR, ROI (large, prominent display)
\item Optimal sizing table (technology × capacity)
\item CAPEX breakdown pie chart
\item Annual revenue breakdown pie chart
\item Quick actions: download report, export Excel, compare scenarios
\end{itemize}

\textbf{FR-WEB-010: Economic Analysis Tab}

Detailed economic results:
\begin{itemize}
\item Cash flow table (annual breakdown)
\item NPV calculation details
\item Payback period chart (cumulative cash flow over time)
\item Cost and revenue breakdowns (stacked bar charts)
\end{itemize}

\textbf{FR-WEB-011: Operational Profiles Tab}

Visualization of typical operation:
\begin{itemize}
\item Dropdown: select representative day (winter/summer/shoulder)
\item Hourly stacked area chart: generation, storage, grid, load
\item Battery SOC line chart (separate axis)
\item Price background color (visual indication of high/low prices)
\item Zoom and pan controls
\end{itemize}

\textbf{FR-WEB-012: Sensitivity Analysis Tab}

Interactive sensitivity exploration:
\begin{itemize}
\item Dropdown: select parameter to vary
\item Slider: adjust parameter value
\item Live update of NPV, IRR, optimal sizing
\item Comparison to baseline case
\end{itemize}

\subsubsection{Scenario Comparison}

\textbf{FR-WEB-013: Comparison Mode (Critical Requirement)}

Side-by-side scenario comparison:
\begin{itemize}
\item Select multiple completed optimizations (checkboxes)
\item Click "Compare" button
\item Comparison view shows:
  \begin{itemize}
  \item Side-by-side metrics table
  \item Comparative bar charts (NPV, IRR, CAPEX)
  \item Sizing comparison (grouped bar chart)
  \item Recommendation: highlight best scenario by selected metric
  \end{itemize}
\end{itemize}

\textbf{FR-WEB-014: Scenario Naming and Organization}

Users can:
\begin{itemize}
\item Name scenarios descriptively ("PV only", "PV + Battery", "With aFRR")
\item Add notes/tags to scenarios
\item Group scenarios by project
\item Filter and search scenarios
\end{itemize}

\subsubsection{Reporting and Export}

\textbf{FR-WEB-015: PDF Report Generation}

"Generate Report" button creates professional PDF:
\begin{itemize}
\item Report template selection (brief/detailed)
\item Include/exclude sections (checkboxes)
\item Company logo upload (branding)
\item Preview before download
\end{itemize}

\textbf{FR-WEB-016: Excel Export}

"Export to Excel" downloads workbook with:
\begin{itemize}
\item Summary sheet
\item Hourly data sheet
\item Cash flow sheet
\item Charts embedded (if possible)
\end{itemize}

\textbf{FR-WEB-017: Chart Downloads}

Each chart has download button:
\begin{itemize}
\item PNG format (for presentations)
\item SVG format (for documents)
\item Interactive HTML (standalone file)
\end{itemize}

\subsubsection{User Experience}

\textbf{FR-WEB-018: Responsive Design}

Interface works on:
\begin{itemize}
\item Desktop browsers (primary target)
\item Tablets (acceptable usability)
\item Mobile (view results only, not configuration)
\end{itemize}

\textbf{FR-WEB-019: Help and Documentation}

Context-sensitive help:
\begin{itemize}
\item "?" icons next to parameters → tooltip explanation
\item "Help" sidebar with relevant documentation section
\item Video tutorials for common workflows
\item Link to full documentation
\end{itemize}

\textbf{FR-WEB-020: Error Handling}

User-friendly error messages:
\begin{itemize}
\item Clear description of problem
\item Suggestion for resolution
\item Option to contact support
\item No technical stack traces shown to end users
\end{itemize}

\subsection{System Integration Requirements}

This subsection describes cross-cutting requirements affecting multiple components.

\subsubsection{Data Persistence}

\textbf{FR-SYS-001: Project Data Storage}

System shall persist:
\begin{itemize}
\item Project definitions (site configuration, devices)
\item Uploaded time series data (load profiles, prices)
\item Optimization job metadata (submission time, status, parameters)
\item Results (optimal sizing, economic metrics, hourly data)
\end{itemize}

Users can retrieve historical results indefinitely.

\textbf{FR-SYS-002: Data Privacy}

Each user/organization's data is isolated:
\begin{itemize}
\item No cross-user data access
\item Secure API authentication
\item Optional data encryption at rest
\end{itemize}

\subsubsection{Performance and Scalability}

\textbf{FR-SYS-003: Concurrent Optimizations}

System shall handle multiple simultaneous optimization jobs:
\begin{itemize}
\item Job queue with priority scheduling
\item Parallel execution on available compute resources
\item Fair scheduling across users
\end{itemize}

\textbf{FR-SYS-004: Large Time Series Handling}

Efficiently handle 20-year hourly time series (175,200 data points):
\begin{itemize}
\item Compress time series for storage
\item Stream large results rather than loading entirely in memory
\item Optional downsampling for visualization
\end{itemize}

\subsubsection{Monitoring and Logging}

\textbf{FR-SYS-005: Optimization Logging}

For each optimization job, log:
\begin{itemize}
\item Input parameters
\item Solver selection and configuration
\item Optimization runtime
\item Solution status (optimal, suboptimal, infeasible)
\item Error messages (if failed)
\end{itemize}

\textbf{FR-SYS-006: System Health Monitoring}

Monitor system health:
\begin{itemize}
\item API response times
\item Optimization queue length
\item Solver availability
\item Error rates
\end{itemize}

Alert administrators if problems detected.

\subsubsection{Deployment Flexibility}

\textbf{FR-SYS-007: Cloud Deployment}

Support deployment on cloud platforms:
\begin{itemize}
\item Containerized architecture (Docker)
\item Horizontal scaling (add compute nodes)
\item Load balancing for API
\item Cloud storage for data persistence
\end{itemize}

\textbf{FR-SYS-008: On-Premises Deployment}

Support on-premises deployment for data security:
\begin{itemize}
\item Single-server installation option
\item Local database
\item No external dependencies (except PVGIS, optional)
\end{itemize}

\textbf{FR-SYS-009: Disconnected Environment Deployment}

Support deployment in secure, disconnected environments:
\begin{itemize}
\item No internet access required after installation
\item PVGIS data pre-cached for relevant locations
\item Local documentation
\end{itemize}
